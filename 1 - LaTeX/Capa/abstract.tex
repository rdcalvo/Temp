Fly-by-wire flight control systems have been supporting innovations in the aircraft industry by providing envelope protection, tailored control laws and maintainability. To perform surface position control, most fly-by-wire systems rely on electro-hydraulic servo actuators and one of their design requirements is the dynamic stiffness.Traditionally, dynamic stiffness requirements are met by either mass balancing of the control surface or by increasing the piston area of hydraulic actuators but both solutions increase aircraft weight. Previous investigations on alternatives to these approaches demonstrated that with an adequate closed-loop system design it is possible to increase the dynamic stiffness of an aerodynamic control surface actuator positioning system, improving the systems' flutter rejection characteristics without jeopardizing the performance requirements usually demanded in this application, allowing a revision of the design characteristics in favor of the global system. However, the design of control loop to increase dynamic stiffness demands valuable resources. To overcome this, optimization techniques can be used since they allow the investigation of a vast number of solutions in much less time than conventional processes. Also, they help engineers understand the effects of parameters and constraints of models and use this information to make better informed decisions. The objectives of this work are to use optimization techniques in hydraulic actuator design to reduce control loop design cycles, maximize dynamic stiffness response and comply with time and frequency domain performance requirements.