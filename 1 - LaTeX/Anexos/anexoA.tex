\section{Control surface flutter suppression requirements}

\begin{quotation}
\begin{scriptsize}

\textbf{25.629   Aeroelastic stability requirements.}

(a) General. The aeroelastic stability evaluations required under this section include flutter, divergence, control reversal and any undue loss of stability and control as a result of structural deformation. The aeroelastic evaluation must include whirl modes associated with any propeller or rotating device that contributes significant dynamic forces. Compliance with this section must be shown by analyses, wind tunnel tests, ground vibration tests, flight tests, or other means found necessary by the Administrator.

(b) Aeroelastic stability envelopes. The airplane must be designed to be free from aeroelastic instability for all configurations and design conditions within the aeroelastic stability envelopes as follows:

(1) For normal conditions without failures, malfunctions, or adverse conditions, all combinations of altitudes and speeds encompassed by the $V_D$/$M_D$ versus altitude envelope enlarged at all points by an increase of 15 percent in equivalent airspeed at both constant Mach number and constant altitude. In addition, a proper margin of stability must exist at all speeds up to $V_D$/$M_D$ and, there must be no large and rapid reduction in stability as VD/MD is approached. The enlarged envelope may be limited to Mach 1.0 when $M_D$ is less than 1.0 at all design altitudes, and

(2) For the conditions described in �25.629(d) below, for all approved altitudes, any airspeed up to the greater airspeed defined by;

(i) The $V_D$/$M_D$ envelope determined by �25.335(b); or,

(ii) An altitude-airspeed envelope defined by a 15 percent increase in equivalent airspeed above $V_C$ at constant altitude, from sea level to the altitude of the intersection of 1.15 $V_C$ with the extension of the constant cruise Mach number line, $M_C$, then a linear variation in equivalent airspeed to $M_C+.05$ at the altitude of the lowest $V_C$/MC intersection; then, at higher altitudes, up to the maximum flight altitude, the boundary defined by a .05 Mach increase in $M_C$ at constant altitude.

(c) Balance weights. If concentrated balance weights are used, their effectiveness and strength, including supporting structure, must be substantiated.

(d) Failures, malfunctions, and adverse conditions. The failures, malfunctions, and adverse conditions which must be considered in showing compliance with this section are:

(1) Any critical fuel loading conditions, not shown to be extremely improbable, which may result from mismanagement of fuel.

(2) Any single failure in any flutter damper system.

(3) For airplanes not approved for operation in icing conditions, the maximum likely ice accumulation expected as a result of an inadvertent encounter.

(4) Failure of any single element of the structure supporting any engine, independently mounted propeller shaft, large auxiliary power unit, or large externally mounted aerodynamic body (such as an external fuel tank).

(5) For airplanes with engines that have propellers or large rotating devices capable of significant dynamic forces, any single failure of the engine structure that would reduce the rigidity of the rotational axis.

(6) The absence of aerodynamic or gyroscopic forces resulting from the most adverse combination of feathered propellers or other rotating devices capable of significant dynamic forces. In addition, the effect of a single feathered propeller or rotating device must be coupled with the failures of paragraphs (d)(4) and (d)(5) of this section.

(7) Any single propeller or rotating device capable of significant dynamic forces rotating at the highest likely overspeed.

(8) Any damage or failure condition, required or selected for investigation by �25.571. The single structural failures described in paragraphs (d)(4) and (d)(5) of this section need not be considered in showing compliance with this section if;

(i) The structural element could not fail due to discrete source damage resulting from the conditions described in �25.571(e), and

(ii) A damage tolerance investigation in accordance with �25.571(b) shows that the maximum extent of damage assumed for the purpose of residual strength evaluation does not involve complete failure of the structural element.

(9) Any damage, failure, or malfunction considered under ��25.631, 25.671, 25.672, and 25.1309.

(10) Any other combination of failures, malfunctions, or adverse conditions not shown to be extremely improbable.

(e) Flight flutter testing. Full scale flight flutter tests at speeds up to $V_{DF}$/$M_{DF}$ must be conducted for new type designs and for modifications to a type design unless the modifications have been shown to have an insignificant effect on the aeroelastic stability. These tests must demonstrate that the airplane has a proper margin of damping at all speeds up to VDF/MDF, and that there is no large and rapid reduction in damping as $V_{DF}$/$M_{DF}$, is approached. If a failure, malfunction, or adverse condition is simulated during flight test in showing compliance with paragraph (d) of this section, the maximum speed investigated need not exceed $V_{FC}$/$M_{FC}$ if it is shown, by correlation of the flight test data with other test data or analyses, that the airplane is free from any aeroelastic instability at all speeds within the altitude-airspeed envelope described in paragraph (b)(2) of this section.
\end{scriptsize}
\end{quotation}

\section{Safety requirements - Flight Control System}

\begin{quotation}
\begin{scriptsize}
25.671   General.

(a) Each control and control system must operate with the ease, smoothness, and positiveness appropriate to its function.

(b) Each element of each flight control system must be designed, or distinctively and permanently marked, to minimize the probability of incorrect assembly that could result in the malfunctioning of the system.

(c) The airplane must be shown by analysis, tests, or both, to be capable of continued safe flight and landing after any of the following failures or jamming in the flight control system and surfaces (including trim, lift, drag, and feel systems), within the normal flight envelope, without requiring exceptional piloting skill or strength. Probable malfunctions must have only minor effects on control system operation and must be capable of being readily counteracted by the pilot.

(1) Any single failure, excluding jamming (for example, disconnection or failure of mechanical elements, or structural failure of hydraulic components, such as actuators, control spool housing, and valves).

(2) Any combination of failures not shown to be extremely improbable, excluding jamming (for example, dual electrical or hydraulic system failures, or any single failure in combination with any probable hydraulic or electrical failure).

(3) Any jam in a control position normally encountered during takeoff, climb, cruise, normal turns, descent, and landing unless the jam is shown to be extremely improbable, or can be alleviated. A runaway of a flight control to an adverse position and jam must be accounted for if such runaway and subsequent jamming is not extremely improbable.

(d) The airplane must be designed so that it is controllable if all engines fail. Compliance with this requirement may be shown by analysis where that method has been shown to be reliable.

\end{scriptsize}
\end{quotation}
